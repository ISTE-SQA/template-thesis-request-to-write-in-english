\documentclass[DIN, parskip=half,% pagenumber=botcenter
               fromalign=left, fromphone=false,%  
               fromemail=true, fromurl=false, %
               fromlogo=false, fromrule=true]{scrlttr2}

\usepackage[utf8]{inputenc}
\usepackage[T1]{fontenc}
\usepackage{lmodern}
\usepackage{times}
\usepackage{ngerman}
\usepackage{tabularx}
\RequirePackage{graphicx}

\setkomavar{fromname}{\sffamily Max Mustermann}
\setkomavar{fromaddress}{\sffamily Musterstraße 42, 12345 Musterstadt}
\setkomavar{fromemail}{\sffamily max@mustermann.de}
\setkomavar{signature}{Max Mustermann}
%\setkomavar{fromlogo}{\includegraphics*[width=2cm]{logo}}

\begin{document}
% If you want headings on subsequent pages,
% remove the ``%'' on the next line:
% \pagestyle{headings}

\begin{letter}{
Institut für Software Engineering\\
z. Hd. Dr. André van Hoorn\\
Universitätsstraße 38\quad\\[0.5cm]
70596 Stuttgart%
}

\opening{
\begin{large}
\textbf{Antrag auf Genehmigung der Masterarbeit in englischer Sprache}
\end{large}\\[0.5cm]
Sehr geehrter Dr. van Hoorn,}

hiermit beantrage ich die Genehmigung meine Masterarbeit mit dem Titel\\
\textbf{\glqq{}Masterarbeitstitel\grqq{}}\\
in englischer Sprache verfassen zu dürfen.

Ich willige gemäß §23, Absatz 6 der Prüfungsordnung für den Masterstudiengang Softwaretechnik (Stand
12. Juli 2012) ein, zusätzlich zur englischsprachigen Zusammenfassung (\glqq{}Abstract\grqq{}) im Anhang der Arbeit eine
Zusammenfassung in deutscher Sprache beizufügen.


% \enlargethispage{3cm}


\closing{Mit freundlichen Grüßen}


\end{letter}
\end{document}
